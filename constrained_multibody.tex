\documentclass{article}

\usepackage{amsmath} 
\usepackage{amsfonts}
\usepackage{mathtools} 
\usepackage{amssymb}
\usepackage{bm}
\usepackage{geometry}
\usepackage{cite}
\usepackage[english]{babel}
\usepackage{amsthm}

% Math Shortcuts: Vector Font: For 3D vectors.
\newcommand{\vf}[1]{{\mf{#1}}}
% Matrix Font: for matrices, arrays and concatenation of 3D vectors.
\newcommand{\mf}[1]{{\mathbf{#1}}}

% Bold greek symbols:
\newcommand{\bgamma}{{\bm\gamma}} \newcommand{\btgamma}{{\mf{\tilde\gamma}}}
\newcommand{\bsigma}{{\bm\sigma}}
\newcommand{\bxi}{{\bm\xi}}
\newcommand{\btheta}{{\bm\theta}}


% Nice customizable remark boxes.
\usepackage[framemethod=TikZ]{mdframed}
\newmdenv[
  linewidth=1pt,
  linecolor=gray,
  backgroundcolor=gray!10,
  roundcorner=5pt,
  skipabove=10pt,
  skipbelow=10pt,
  innertopmargin=6pt,
  innerbottommargin=6pt,
  innerleftmargin=8pt,
  innerrightmargin=8pt,
  nobreak=true   % <-- prevents page breaks inside
]{remarkbox}
\newenvironment{remark}{\begin{remarkbox}\textbf{Remark.}~}{\end{remarkbox}}

\geometry{margin=1in}

\begin{document}

\begin{center}
    {\LARGE \textbf{Notes on Constrained Multibody Dynamics}}\\[0.5em]
    {\large Alejandro Castro}\\[0.5em]
    {\large April 1, 2025}
\end{center}

\section{Preliminaries}

We consider a mechanical system with generalized positions $\mf{q} \in
\mathbb{R}^{n_q}$ and generalized velocities $\mf{v} \in \mathbb{R}^{n_v}$,
where $n_q$ and $n_v$ are the number of generalized positions and velocities,
respectively.

The system evolves according to
\cite{bib:featherstone2008_rigid_body_dynamics_algorithms,bib:jain2010_robot_multibody_dynamics}:

\begin{eqnarray}
    &\mf{M}(\mf{q}) \dot{\mf{v}} + \mf{c}(\mf{q}, \mf{v}) = \bm{\tau},
    \label{eq:momentum_balance}\\    
    &\dot{\mf{q}} = \mf{N}(\mf{q})\mf{v},
    \label{eq:kinematic_map}
\end{eqnarray}

where $\mf{M}(\mf{q}) \in \mathbb{R}^{{n_v} \times {n_v}}$ is the mass matrix,
$\mf{c}(\mf{q}, \mf{v}) \in \mathbb{R}^{n_v}$ includes Coriolis and centrifugal
effects, $\bm{\tau} \in \mathbb{R}^{n_v}$ are generalized forces, and
$\mf{N}(\mf{q})$ is the kinematic map that relates time derivatives of the
generalized positions to generalized velocities \cite{bib:diebel2006}.

\begin{remark}
The above equations are a nice compact way to describe the dynamics of a general
system, including articulated bodies (say robots or even ships with their
propellers and rudders).
When only a single rigid body is considered, the mass matrix boils down to something like:
\begin{equation*}
    \mf{M}(\mf{q}) = \begin{bmatrix}
        \mf{I}(\mf{q}) & \vf{p}(\mf{q})_\times \\
        \vf{p}(\mf{q})_\times^T & m \mathbf{I}_3
        \end{bmatrix}
\end{equation*}
where, for an inertia computed about a center point $B_o$, $\vf{p}$ is the
position vector of the center of mass from $B_o$ (zero if $B_o$ is the CoM), $m$
is the body mass, $\mf{I}(\mf{q})$ is the rotational inertia about $B_o$ (where
each component corresponds to the moments and products of inertia).

For this case the generalized forces $\bm{\tau}$ correspond to the torques and
moments on the rigid body.
\end{remark}  

\begin{remark}
For the typical case where $B_o$ is the CoM and we measure velocities in an
inertial frame, the expression for $\mf{c}(\mf{q}, \mf{v})$ can be found in
\cite[\S 2.3.1]{bib:jain2010_robot_multibody_dynamics}.
\end{remark}

\begin{remark}
There are many ways to represent orientations. The two most common ones are
quaternions and Cardan angles (roll-pitch-yaw). Similarly, for generalized
velocities the two most common choices are angular velocities expressed in the
world or in the body frame (I personally prefer world frame, though body-frame
is extremely common). Whichever the choice, we can find the angular component of
the kinematics map (typically the translational part trivially will be identity)
in \cite[\S 5.6]{bib:diebel2006} for Cardan angles and in \cite[\S
6.6]{bib:diebel2006} for quaternions.
\end{remark}

We wish to integrate the system
\eqref{eq:momentum_balance}-\eqref{eq:kinematic_map} forward in time, subject to
holonomic constraints described as:

\begin{equation}
    \mf{\Phi}(\mf{q}, t) = \mf{0},
\end{equation}

such that $\mf{\Phi}: \mathbb{R}^{n_q} \times \mathbb{R} \rightarrow \mathbb{R}^k$
imposes $k$ constraint equations on the dynamics of the system.

\begin{remark}
As an example, consider the experiment of a model scale ship in a towing tank,
held by mount that imposes the ship to a specified heave $h(t)$, and allows it
to pitch freely. All other motions are constrained.

For this case, the constraint function will be:
\begin{eqnarray*}
    \mf{\Phi}(\mf{q}, t) = \begin{bmatrix}
        \phi\\
        \psi\\
        x\\
        y\\
        z - h(t)
    \end{bmatrix}  
\end{eqnarray*}
where, as generalized coordinates I am using $\mf{q}=[\phi, \theta, \psi,
x, y, z]^T$, with $\phi, \theta, \psi$ the roll, pitch and yaw angles,
respectively.

\end{remark}    

\section{Constrained Dynamics at the Acceleration Level}

Differentiating the constraint (twice) and enforcing it at the acceleration
level gives:

\begin{equation}
    \mf{J}(\mf{q}) \dot{\mf{v}} + \mf{b}(\mf{q},\mf{v},t) = \mf{0}
\end{equation}

where \( \mf{J} = \partial \mf{\Phi}/\partial\mf{q}\,\mf{N}(\mf{q})\) is the
constraint Jacobian and $\mf{b}(\mf{q},\mf{v},t)$ is the acceleration bias.

\begin{remark}
From now on I'll consider the most common case of a separable constraint of the
form $\mf{\Phi}(\mf{q}, t) = \mf{\Phi}_q(\mf{q}) + \mf{\Phi}_t(t)$. For this
case $\mf{J}(\mf{q}) = \partial\mf{\Phi}_q/\partial\mf{q}\,\mf{N}(\mf{q})$ and
$\mf{b}= \dot{\mf{J}}\mf{v}+d^2\Phi_t/dt^2$.
\end{remark}

We enforce the constraints using Lagrange multipliers \( \mf{\lambda} \in
\mathbb{R}^k \), resulting in the constraint dynamics:

\begin{eqnarray}
    &\mf{M}(\mf{q}) \dot{\mf{v}} + \mf{c}(\mf{q}, \mf{v}) = \bm{\tau} + \mf{J}^T(\mf{q}) \mf{\lambda},\\    
    &\dot{\mf{q}} = \mf{N}(\mf{q})\mf{v},\\
    &\mf{J}(\mf{q}) \dot{\mf{v}} + \mf{b}(\mf{q},\mf{v},t) = \mf{0},\\
    &\mf{\Phi}(\mf{q}, t) = \mf{0}.
\end{eqnarray}

While these equations describe the continuous-time evolution of the system, they
are not quite useful for numerical integration. For numerical integration we
usually formulate the same equations at the velocity level.

\section{Discrete Velocity-Level Formulation}

To decouple integration and constraint enforcement, we use an operator splitting
strategy.

Let \( \mf{v}^* \) be the unconstrained velocity obtained using any integration
scheme applied to:

\begin{equation}
    \mf{M}(\mf{q}) \dot{\mf{v}} + \mf{c}(\mf{q}, \mf{v}) = \bm{\tau}.
\end{equation}
Say for instance, we use RK4 or Crank-Nicolson to compute $\mf{v}^*$.

We then correct the velocity to satisfy the constraints by solving:

\begin{equation}
    \mf{M}(\mf{q}^n)(\mf{v}^{n+1} - \mf{v}^*) = \mf{J}^T(\mf{q}^n) \mf{\lambda}^{n+1},
    \label{eq:discrete_momentum}
\end{equation}
where now $\mf{\lambda}^{n+1}$ contain impulses (forces times time step $h$).

We obtain the constraint at the velocity level by differentiating it once:
\begin{equation}
    \mf{J}^n\mf{v}^{n+1} = -\frac{\partial\mf{\Phi}}{\partial t}.
    \label{eq:velocity_constraint}
\end{equation}

We can solve for the Lagrange multipliers by substitution of
\eqref{eq:velocity_constraint} into \eqref{eq:discrete_momentum}:
\begin{equation}
    \mf{\lambda}^{n+1} = -(\mf{J}\mf{M}\mf{J}^T)^{-1}(\frac{\partial\mf{\Phi}}{\partial t} + \mf{J}\mf{v}^*).
\end{equation}

We then use $\mf{\lambda}^{n+1}$ in \eqref{eq:discrete_momentum} to compute
${\mf{v}}^{n+1}$. With the velocities, we can now advance the configuration to
${\mf{q}}^{n+1}$.

\begin{remark}
 Equation \eqref{eq:discrete_momentum} opens the door to a great modular way to
 reuse your code and update it to incorporate constraints. You can use your
 existing code to obtain $\mf{v}^*$. Then Lagrange multipliers can be used to
 compute $\mf{v}^{n+1}$ from $\mf{v}^*$, plus the \emph{projection} method below
 to mitigate constraint drift.
 \end{remark}

\section{Constraint Stabilization}

\subsection{Projection}

Numerical integration of the dynamics often causes constraint drift, where \(
\mf{\Phi}(\mf{q}, t) \neq 0 \) due to accumulated numerical errors. The
projection method corrects this drift by mapping the configuration back onto the
constraint manifold \cite{bib:cline2002rigid}.

Let \( \mf{q}^* \) be a tentative configuration (computed by advancing the
dynamics forward in time as in the previous section) that does not satisfy the
constraint, i.e. \( \mf{\Phi}(\mf{q}^*, t) = \mf{\varepsilon} \neq \mf{0} \).
The goal is to find a corrected configuration \( \mf{q}_{\text{proj}} \) such
that:

\begin{equation}
    \mf{\Phi}(\mf{q}_{\text{proj}}, t) = \mf{0},
\end{equation}

and \( \mf{q}_{\text{proj}} \) is as close as possible to \( \mf{q}^* \). This
is a constrained optimization problem:

\begin{eqnarray}
    &\mf{q}_{\text{proj}} = \arg\min_{\mf{q}} \frac{1}{2} \| \mf{q} - \mf{q}^* \|^2 \\
    &\text{s.t.} \quad \mf{\Phi}(\mf{q}, t) = \mf{0}.
    \label{eq:projection}
\end{eqnarray}

\subsubsection*{Linearization and Newton Iteration}

Assuming \( \mf{q}^* \) is close to the constraint manifold, we perform a
first-order Taylor expansion of the constraint:

\begin{equation}
    \mf{\Phi}(\mf{q}_{\text{proj}}, t) \approx \mf{\Phi}(\mf{q}^*, t) + \mf{J}_q^*\Delta
    \mf{q} = \mf{0},
\end{equation}
with $\mf{J}_q^*=\frac{\partial \mf{\Phi}}{\partial \mf{q}}(\mf{q}^*)$ and $\Delta
\mf{q} = \mf{q}_{\text{proj}} - \mf{q}^*$.

\begin{equation}
    \mf{J}_q^* \Delta \mf{q} = -\mf{\Phi}^*.
\end{equation}

With this linear approximation the non-linear optimization in
\eqref{eq:projection} becomes the quadratic program (QP):
\begin{eqnarray}
    \Delta\mf{q} = \arg\min_{\mf{q}} \frac{1}{2} \| \Delta\mf{q} \|^2 \\
    \text{s.t.} \quad \mf{J}_q^*\Delta\mf{q} = -\mf{\Phi}^*.
    \label{eq:linear_projection}
\end{eqnarray}

This can be solved using Lagrange multipliers. The result is:
\begin{equation}
    \Delta \mf{q} = -\mf{J}_q^T(\mf{J}_q \mf{J}_q^T)^{-1} \mf{\Phi}(\mf{q}^*, t),
\end{equation}

When constraints are independent and not overdetermined $\mf{J}_q \mf{J}_q^T$ is
invertible. When this is not true, we can compute the least-squares solution
using the pseudoinverse of $\mf{J}_q \mf{J}_q^T$. This operation projects \(
\mf{q}^* \) orthogonally onto the constraint surface, in the Euclidean sense.

\subsubsection*{Iterative Newton-Raphson Correction}

The linearized update is only locally accurate. To enforce the constraint up to
a desired tolerance, we apply the correction iteratively. At each iteration \( k
\), we compute:

\begin{align}
    \mf{\Phi}^k &= \mf{\Phi}(\mf{q}^k, t), \\
    \mf{J}_q^k &= \frac{\partial \mf{\Phi}}{\partial \mf{q}} (\mf{q}^k, t), \\
    \Delta \mf{q}^k &= -{\mf{J}_q^k}^T (\mf{J}_q^k {\mf{J}_q^k}^T)^{-1} \mf{\Phi}^k, \\
    \mf{q}_{k+1} &= \mf{q}^k + \Delta \mf{q}^k.
\end{align}

The iteration continues until \( \| \mf{\Phi}(\mf{q}^k, t) \| < \epsilon \) for
some small threshold \( \epsilon \). This Newton-based projection ensures that
\( \mf{q}_{\text{proj}} \) lies on the constraint manifold up to numerical
precision.

\begin{remark}    
\begin{itemize}
    \item The method assumes \( \mf{J}_q \mf{J}_q^T \) is invertible. This
    requires that the constraints be independent and not overdetermined. We can
    use the pseudoinverse of $\mf{J}_q \mf{J}_q^T$ to obtaine the least-squares
    solution.
    \item The projection can be
    modified to minimize in the kinetic energy norm \cite{bib:cline2002rigid}
    using the generalized mass matrix \( \mf{M} \) . This leads to:
    \[
        \Delta \mf{q} = -\mf{M}^{-1} \mf{J}_q^T (\mf{J}_q \mf{M}^{-1} \mf{J}_q^T)^{-1} \mf{\Phi}(\mf{q}^*, t),
    \]
    which corresponds to the least kinetic energy correction.
\end{itemize}
\end{remark}

\subsection{Baumgarte Stabilization}

This is an alternative to the projection method to correct for constraint drift.
We enforce constraints at the velocity level with proportional-derivative
feedback:

\begin{equation}
    \dot{\mf{\Phi}} + \frac{1}{\tau}\mf{\Phi} = \mf{0},
\end{equation}
where $\tau$ is a time constant usually chosen to be a factor of the time step
size $h$.

We use this equation in terms of velocities to replace \eqref{eq:velocity_constraint}:
\begin{equation}
    \mf{J}^n\mf{v}^{n+1} = -\frac{\partial\mf{\Phi}}{\partial t} - \frac{1}{\tau}\mf{\Phi}.
\end{equation}

\begin{remark}
This is a simple method when compared to projection. However choosing $\tau$
requires tuning and it can make the system unstable. Moreover, this method
effectively allows small deviations from a perfect constraint. In other words,
we added numerical compliance.
\end{remark}

Below I present what I think it's the simplest method to implement. It does not
require Lagrange multipliers, and constraint drift is controlled in terms of
physically compliant forces.

\section{Penalty-Based Methods (Soft Constraints)}

Constraints are enforced approximately by using a model of compliance, where the
Lagrange multipliers follow a spring-damper law:
\begin{equation}
    \bm{\lambda} = -h(k_p \mf{\Phi}(\mf{q}, t) + k_d \dot{\mf{\Phi}}(\mf{q}, t)),
\end{equation}
where the time step $h$ is needed for impulses instead of forces in a
velocity-level formulation.

with \( k_p \) and \( k_d \) being stiffness and damping coefficients,
respectively. Different stiffness and damping can be used for each constraint.
Typically there will be a an easy physical interpretation of these equations.
For instance, for the towing tank experiment example above, those coefficients
would model a compliant (not rigid) mounting point. Very hight stiffness values
can be used in practice without affecting stability (we must of course use
implicit methods).


We now use the approximations:
\begin{eqnarray}
    \mf{q}^{n+1} \approx \mf{q}^{n} + h \mf{N}^n \mf{v}^{n+1},\\
    \mf{\Phi}^{n+1} \approx \mf{\Phi}^{n} + h \mf{J}^n \mf{v}^{n+1},\\
    \dot{\mf{\Phi}}^{n+1} \approx \mf{J}^n \mf{v}^{n+1} + \frac{\partial\mf{\Phi}}{\partial t}^{n+1}.
\end{eqnarray}

With these approximations, we can write the (compliant) Lagrange multipliers
implicitly in the next time step generalized velocity $\mf{v}^{n+1}$:
\begin{equation}
    \bm{\lambda} = -h(k_p + h k_d) \mf{J}^n \mf{v}^{n+1} - h\left(k_p\mf{\Phi}^{n}+k_d\frac{\partial\mf{\Phi}}{\partial t}^{n+1}\right).
\end{equation}

We can now incorporate these forces implicitly in \eqref{eq:discrete_momentum},
solve for the next time step velocities, and advance configurations:
\begin{eqnarray}
    \mf{v}^{n+1} = \left(\mf{M}^n + h(k_p + h k_d){\mf{J}^n}^T\mf{J}^n\right)^{-1}\left(\mf{M}^n\mf{v}^*-h\left(k_p\mf{\Phi}^{n}+k_d\frac{\partial\mf{\Phi}}{\partial t}^{n+1}\right)\right),
    \label{eq:compliant_momentum}\\
    \mf{q}^{n+1} = \mf{q}^{n} + h \mf{N}^n \mf{v}^{n+1}.
    \label{eq:configuraiton_update}
\end{eqnarray}

Notice that, unlike with projection methods, matrix $\mf{M}^n + h(k_p + h
k_d){\mf{J}^n}^T\mf{J}^n$ is invertible always, given that the mass matrix is
positive definite and ${\mf{J}^n}^T\mf{J}^n$ is semi positive definite. This
property is expected given the physical nature of the compliant forces (while
rigid constraints are just an approximation of reality).

\begin{remark}
In case it wasn't clear, equations
\eqref{eq:compliant_momentum}-\eqref{eq:configuraiton_update} is what I
recommend as a first (final?) pass. I decided to include projection methods for
completeness and to introduce notation. Also so that you could see how much more
complex they are to implement for no additional gain really. 

In contrast, compliant forces are very easy to implement and work great in
practice. Large values of stiffness and damping do not make the numerics harder,
especially so for a system of a single rigid body.
\end{remark}

\begin{remark}
You can always implement a projection method at a later stage if desired. The
only advantage if any, might be that thy are parameter free.
\end{remark}    

\bibliographystyle{unsrt}  % or 'alpha', 'ieeetr', etc.
\bibliography{constrained_multibody}  % assumes your .bib file is named references.bib

\end{document}

