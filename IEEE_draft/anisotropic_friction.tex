\section{Anisotropic Coulomb Friction}
\label{sec:anisotropic_friction}

When friction is
anisotropic, the friction force can have a component perpendicular to the slip
velocity and, even in the absence of external forces, objects follow curved
paths~\cite{bib:walker2019}. The ellipsoidal friction cone
$\mathcal{F}=\{[\vf{x}_t,
x_n]\in\mathbb{R}^3\,|\,\Vert\vf{\mu}^{-1}\vf{x}_t\Vert \le x_n\}$, with
$\vf{\mu}\succeq 0$ the friction tensor, is a popular approximation. $\vf{\mu}$
is diagonal when expressed in a frame aligned with its principal axes, and
$\vf{\mu} = \mu\mf{I}$ for isotropic friction.

We write an anisotropic model of Coulomb friction that satisfies
the principle of maximum dissipation
\begin{equation}
    \bgamma_{t}=\argmax_{\bxi\in\mathcal{F}}-\vf{v}_{t}\cdot\bxi.
    \label{eq:maximum_dissipation_principle}
\end{equation}

We can solve this problem analytically by a change of variables that maps the
ellipsoidal section of the friction cone into a circular section. The result is
\begin{equation}
    \bgamma_{t}=-\vf{\mu}\gamma_n\hat{\vf{t}}(\vf{\mu}\vf{v}_{t}),
    \label{eq:anisotropic_friction_model}
\end{equation}
where we define
$\hat{\vf{t}}(\vf{\mu}\vf{v}_{t})=\vf{\mu}\vf{v}_{t}/\Vert\vf{\mu}\vf{v}_{t}\Vert$.
This friction model opposes slip when tensor $\vf{\mu}$ is isotropic, while it
introduces a component perpendicular to the line of motion, as
experimentally confirmed in~\cite{bib:walker2019}.

As with~\eqref{eq:generic_friction_model}, we write a generic form of
this model as
\begin{eqnarray}
    \bgamma_{t}=g(\Vert\tilde{\vf{v}}_t\Vert, v_n)\,\tilde{\vf{t}},
    \label{eq:generic_anisotropic_model}
\end{eqnarray}
where we defined the \emph{tilde} quantities as
$\tilde{\vf{v}}_t=\vf{\mu}\,\vf{v}_{t}$ and
$\tilde{\vf{t}}=\vf{\mu}\,\hat{\vf{t}}(\tilde{\vf{v}}_t)$, consistent with
\eqref{eq:anisotropic_friction_model}. We verify that
\begin{equation*}
    \frac{\partial\bgamma_t}{\partial\mf{v}_t}=\frac{\partial g}{\partial \Vert\tilde{\vf{v}}_t\Vert}\tilde{\mf{P}}+g\frac{\tilde{\mf{P}}^\perp}{\Vert\tilde{\vf{v}}_t\Vert},
\end{equation*}
with $\tilde{\mf{P}}=\mf{P}(\tilde{\vf{t}})$ and $\tilde{\mf{P}}^\perp =
\vf{\mu}\mf{P}^\perp(\hat{\vf{t}}(\tilde{\vf{v}}_t))\vf{\mu}=\vf{\mu}^2-\tilde{\mf{P}}$.
Therefore, $\partial\bgamma_t/\partial\mf{v}_t$ for~\eqref{eq:generic_anisotropic_model}
is symmetric and condition~\eqref{eq:curl_normal} is met.

Using these \emph{tilde} variables, we define a potential for Lagged
\begin{equation*}
    \ell_t(\vf{v}_t) = \gamma_{n0}\,\varepsilon_s\,F(\Vert\tilde{\vf{v}}_t\Vert/\varepsilon_s),
\end{equation*}
and a potential for Similar is obtained by updating \eqref{eq:similar_grouping}
to $z=v_n-\varepsilon_sF(\Vert\tilde{\vf{v}}_t\Vert/\varepsilon_s)$.
