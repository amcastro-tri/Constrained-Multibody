\section{Mathematical Formulation}
\label{sec:optimization_framework}

\subsection{Preliminaries}

I'll use notation from the articulated body dynamics community
\cite{bib:featherstone2008_rigid_body_dynamics_algorithms,bib:jain2010_robot_multibody_dynamics},
which is most used in robotics and graphics, simply because it's compact. I'll
highlight important aspects and the specific case for a single rigid body as I
go.

State is described by generalized positions $\mf{q} \in \mathbb{R}^{n_q}$ and
generalized velocities $\mf{v} \in \mathbb{R}^{n_v}$, where $n_q$ and $n_v$ are the
number of generalized positions and velocities, respectively. Time derivatives
of the generalized positions relate to the generalized velocities by the kinematic map
$\dot{\mf{q}} = \mf{N}(\mf{q})\mf{v}$. 

There are several options to choose from to define generalized positions and
velocities for a single rigid body. For instance, to describe spatial
orientation, we can use roll-pitch-yaw angles or quaternions. For generalized
velocities there's notice that $\dot{\mf{q}} = \mf{N}(\mf{q})\mf{v}$ allows for
$\mf{v} \ne \dot{\mf{q}} $. For instance, a very common choice of generalized
velocities to describe rotations is the angular velocitity of the body. There
are published tables that allow to compute $\mf{N}(\mf{q})$ for different
choices of generalized coordinates 



We first introduce notation and conventions used in our framework. We closely
follow the notation in our previous work~\cite{bib:castro2022unconstrained,bib:masterjohn2021discrete},
later extended to deformable Finite Element Method (FEM) models in~\cite{bib:han2023}
and Material Point Method (MPM) models in~\cite{bib:zong2024_convex_mpm}.

State is described by generalized positions $\mf{q} \in \mathbb{R}^{n_q}$ and
generalized velocities $\mf{v} \in \mathbb{R}^{n_v}$, where $n_q$ and $n_v$ are the
number of generalized positions and velocities, respectively. Time derivatives
of the generalized positions relate to the generalized velocities by the kinematic map
$\dot{\mf{q}} = \mf{N}(\mf{q})\mf{v}$. We use joint coordinates to describe
articulated rigid bodies.

Consider a system with $n_c$ constraints introducing $n_\text{eq}$ constraint
equations. Constraint velocities $\mf{v}_c\in\mathbb{R}^{n_\text{eq}}$ are given
by $\mf{v}_c=\mf{J}\mf{v}+\mf{b}$, where $\mf{J}$ is the stacked Jacobian for
all constraints and $\mf{b}$ is a bias term. Following
\cite{bib:castro2022unconstrained}, we write a symplectic Euler scheme with
fixed time steps of size $\delta t$ as
\begin{flalign}
    % Momentum equation.
	&\mf{M}(\mf{q}_0)(\mf{v}-\mf{v}_0) =\nonumber\\
    &\qquad\delta t\,\mf{k}_1(\mf{q},\mf{v}) + \delta t\,\mf{k}_2(\mf{q}_0,\mf{v}_0) +
	\mf{J}(\mf{q}_0)^T\mf{\bgamma},
    \label{eq:scheme_momentum}\\
    &\mf{q} = \mf{q}_0 + \delta t \mf{N}(\mf{q}_0)\mf{v}\nonumber,
\end{flalign}
where quantities with the naught subscript are evaluated at the previous time
step $t^n$ and quantities without subscripts are evaluated at the next time step
$t^{n+1}= t^n+\delta t$. $\mf{M}$ is the joint space mass matrix. Stiff terms,
such as springs, and stabilizing terms, such as damping and rotor inertias, are
treated implicitly in $\mf{k}_1$. Smooth, non-stiff terms such as gravity,
gyroscopic forces, and feed-forward actuation are treated explicitly in
$\mf{k}_2$. Vector $\bgamma\in\mathbb{R}^{n_\text{eq}}$ corresponds to
constraint impulses. Our implementation in Drake~\cite{bib:drake} includes
contact constraints, holonomic constraints (e.g. weld, distance, coupler), and PD
controllers with effort limits \cite{bib:castro2022unconstrained, bib:han2023}.

We define the momentum residual $\mf{m}(\mf{v})$ as
\begin{equation*}
    \mf{m}(\mf{v}) = \mf{M}(\mf{q}_0)(\mf{v}-\mf{v}_0) -
    \delta t\,\mf{k}_1(\mf{q},\mf{v}) - \delta t\,\mf{k}_2(\mf{q}_0,\mf{v}_0),
\end{equation*}
and define the \emph{free motion velocities} $\mf{v}^*$ as the solution to
$\mf{m}(\mf{v}^*)=\mf{0}$, that is, the velocities the system would have in the
absence of constraint forces. SAP \cite{bib:castro2022unconstrained} linearizes
\eqref{eq:scheme_momentum} at $\mf{v} = \mf{v}^*$ as
\begin{eqnarray}
	\mf{A}(\mf{v}-\mf{v}^*) = \mf{J}^T\bgamma,
    \label{eq:optimality_condition}
\end{eqnarray}
where $\mf{A}$ is a symmetric positive-definite (SPD) approximation of the gradient of the momentum residual
accurate to first order, i.e.
\begin{align*}
	\left. \frac{\partial \mf{m}}{\partial \mf{v}} \right|_{\mf{v}=\mf{v}^*} = \mf{A} + \mathcal{O}(\delta t).
\end{align*}

While early work~\cite{bib:stewart1996implicit, bib:anitescu1997} use $\mf{A} =
\mf{M}(\mf{q}_0)$, the mass matrix at the previous time step, our formulation
incorporates the modeling of joint springs, damping, and rotor inertia
implicitly~\cite{bib:castro2022unconstrained}. 

Traditional NCP formulations supplement~\eqref{eq:optimality_condition} with
constraint equations to model contact. Additional constraint equations, slack
variables, and Lagrange multipliers lead to a large and challenging-to-solve NCP.
Instead, SAP~\cite{bib:castro2022unconstrained} follows a different approach.
Inspired by the analytical inverse dynamics from Todorov~\cite{bib:todorov2011},
SAP eliminates constraints analytically to write an unconstrained convex
optimization problem for the next time step velocities
\begin{eqnarray}
	\min_{\mf{v}} \ell_p(\mf{v}) = \frac{1}{2}\Vert\mf{v}-\mf{v}^*\Vert_{A}^2 +
	\ell_R(\mf{v}),
    \label{eq:unconstrained_fomulation}
\end{eqnarray}
where $\Vert\mf{x}\Vert_{A}^2 = \mf{x}^T\mf{A}\,\mf{x}$, and $\ell_R(\mf{v})$ is
a regularizing term that penalizes contact impulses. It can be shown that
SAP's~\cite{bib:castro2022unconstrained} formulation models linear compliant contact
with a Kelvin-Voigt (linear) model of damping. The formulation is strongly
convex, with convergence guarantees that lead to robust software
implementations. Although the formulation is inherently compliant with
regularized friction, \cite{bib:castro2022unconstrained}~shows how rigid contact
can be modeled to good approximation using a \emph{near-rigid} approximation.

\subsection{Incremental Potentials}

Inspired by \emph{incremental potentials} \cite{bib:li2020ipc,
bib:pandolfi2002}, we extend formulation \eqref{eq:unconstrained_fomulation} by
replacing the \emph{regularizer} cost $\ell_R(\mf{v})$ with a more general
incremental potential $\ell_c(\mf{v}; \mf{x}_0)$ to model contact and other
constraints. This potential depends on the generalized velocities $\mf{v}$,
\emph{given} a previous state $\mf{x}_0$, making it inherently discrete---unlike
continuous potentials like gravity or electric fields. Nonetheless, as we will
discuss, these discrete potentials have real physical relevance. The presence of
the quadratic term $\Vert\mf{x}\Vert_{A}^2$ ensures strong convexity of the
optimization problem \eqref{eq:unconstrained_fomulation} when $\ell_c$ is
convex, guaranteeing a unique solution. Developing such convex potentials for
frictional contact modeling is the primary aim of this work.

Within this framework, the balance of momentum~\eqref{eq:optimality_condition}
is the optimality condition of~\eqref{eq:unconstrained_fomulation}, and impulses
emerge as a result of the contact potentials
\begin{eqnarray}
    \bgamma(\mf{v}_c; \mf{x}_0) = -\frac{\partial\ell_c(\mf{v}_c;
    \mf{x}_0)}{\partial\mf{v}_c},
    \label{eq:gamma_from_potential}
\end{eqnarray}
where we use the notation $\partial f/\partial\vf{x} = \nabla f \in\mathbb{R}^n$
to denote the gradient of a scalar function $f(\vf{x}):
\mathbb{R}\rightarrow\mathbb{R}^n$. As with SAP, we consider potential functions
that are
\emph{separable}
\begin{eqnarray}
    \ell_c(\mf{v}_c; \mf{x}_0) = \sum_{i=1}^{n_c}\ell_{c,i}(\vf{v}_{c,i};
    \mf{x}_0),
    \label{eq:separable_potential}
\end{eqnarray}
where $\mf{v}_c$ is the stacked vector of individual constraint velocities
$\vf{v}_{c,i}$. For contact constraints, we define a contact frame $C_i$ for
which we arbitrarily choose the $z\text{-axis}$ to coincide with the contact
normal $\hat{\vf{n}}_i$. In this frame, the normal and tangential components of
$\vf{v}_{c,i}$ are given by $v_{n,i} = \hat{\vf{n}}_i\cdot\vf{v}_{c,i}$ and
$\vf{v}_{t,i} = \vf{v}_{c,i}-v_{n,i}\hat{\vf{n}}_i$ respectively, and
$\vf{v}_{c,i}=[\vf{v}_{t,i}\,v_{n,i}]$. By convention, we define the relative
contact velocity $\vf{v}_{c,i}$ and normal $\hat{\vf{n}}_i$ such that $v_{n,i} >
0$ for objects moving away from each other.

Using this separable potential \eqref{eq:separable_potential}
in~\eqref{eq:gamma_from_potential}, the impulse vector $\bgamma$ is the stacked
vector of individual constraint contributions $\bgamma_i(\mf{v}_{c,i}; \mf{x}_0)
= -\partial\ell_{c,i}/\partial\mf{v}_{c,i}$. We highlight the dependence
on the previous time step state $\mf{x}_0$ to emphasize the discrete (or
incremental) nature of these potentials and the resulting impulses. However,
hereinafter, we write $\bgamma_i(\vf{v}_{c,i})$ and $\ell_{c,i}(\vf{v}_{c,i})$ to
shorten notation, and the functional dependence on the previous time step state
is implicitly assumed.

\section{Frictionless Contact}
\label{sec:frictionless_contact}

We discuss frictionless contact first to introduce the framework and notation.
We consider a generic normal force model
\begin{equation}
    f_n(\phi, v_n),
    \label{eq:generic_force_law}
\end{equation}
as a function of the \emph{signed distance} $\phi$ (defined negative when
objects overlap) and the normal velocity $v_n$ (defined positive when objects
separate). To write a discrete incremental potential, we use a first order
approximation of the signed distance $\phi=\phi_0+\delta t v_n$, implicit in the
next time step velocity $v_n$. Using this approximation, we define a discrete
impulse as
\begin{equation}
    n(v_n; \phi_0) = \delta t\,f_n(\phi_0+\delta t\,v_n, v_n),
    \label{eq:discrete_normal_impulse}
\end{equation}
and the normal impulse is $\gamma_n(v_n)=n(v_n; \phi_0)$. Therefore, the
potential for such model is given by
\begin{equation*}
    \ell_n(v_n) = -N(v_n),
\end{equation*}
with $N(v_n)$ the antiderivative of $n(v_n)$, i.e. $n(v_n) = N'(v_n)$.

The Hessian of this cost is
\begin{equation*}
    \frac{d^2\ell_n}{dv_n^2} = -\delta t^2\frac{\partial f_n}{\partial\phi} - \delta t\frac{\partial f_n}{\partial v_n},
\end{equation*}
and therefore 
\begin{equation}
    \frac{\partial f_n}{\partial\phi}\le 0\text{ and }
    \frac{\partial f_n}{\partial v_n}\le 0,
    \label{eq:normal_sufficient_conditions}
\end{equation}
are sufficient conditions for the cost $\ell_n(v_n)$ to be convex.

\subsection{Barrier Functions for Rigid Contact}

Interior-point methods (IP) are the standard for solving optimization problems
with inequality constraints, essential for rigid contact modeling. IP solvers
can also handle a richer set of constraints, like conic constraints, as needed
to model Coulomb friction. Notable solvers include open-source
Ipopt~\cite{bib:wachter2006} and proprietary options like
Gurobi~\cite{bib:gurobi} and Mosek~\cite{bib:mosek}. Fundamentally, IP methods
replace constraints with a barrier function that penalizes infeasible solutions.
Logarithmic functions are commonly used, which for rigid contact penalize
distances near zero
\begin{equation*}
    \ell_n(\phi) = -\kappa \ln(\phi),
\end{equation*}
while penetration states ($\phi\le0$) are infeasible. The barrier parameter
$\kappa > 0$ is iteratively reduced to approach a rigid approximation, though it
can never reach zero. Therefore, in practice the solution is effectively
compliant, with a force law that fits the framework
\eqref{eq:generic_force_law}
\begin{equation}
    f_n(\phi, v_n) = \frac{\kappa}{\phi}\quad  \phi \ge 0.
    \label{eq:logarithmic_barrier_force}
\end{equation}

As a physical model of contact, users may struggle to
interpret~\eqref{eq:logarithmic_barrier_force}, which has a parameter $\kappa$
with the unit of energy and nonphysical action at infinite distance. In
practice, $\kappa$ is not exposed, but hidden as part of the solver internals.
Users \emph{believe} they are working with a true model of rigid contact when,
in reality, the solver is using an compliant model approximation. Even if
$\kappa$ can be reduced to very small values (according to some hidden metric),
the solver often ends up solving a much more challenging problem than needed
since, in reality, physical materials have finite stiffness.

\subsection{Incremental Potential Contact}

IPC ~\cite{bib:li2020ipc} is an optimization-based framework to model rigid
contact and guarantee intersection-free solutions. The method attains strong
robustness given its implicit time-stepping scheme and line
search augmented with continuous collision detection (CCD) to maintain
feasibility. However, IPC formulates a non-convex optimization problem and can
fall into local minima, not satisfying the original physical laws. Moreover, the
method lacks convergence guarantees, as properly pointed out by the original
authors.

Similar to the logarithmic barrier functions of IP methods, IPC proposes a
smooth $C^2$ potential
\begin{eqnarray*}
    \ell_c(\phi) = -\kappa_\text{IPC}\,(\hat{d}-\phi)_+^2\ln(\phi/\hat{d})
\end{eqnarray*}
where $(a)_+ = \max(0, a)$, $\kappa_\text{IPC}$ is a parameter automatically
adjusted to improve numerical conditioning, and $\hat{d}$ is a user parameter.
Typical values of $\hat{d}$ used by the original authors in their extensive
simulations test cases are in the range $0.1\text{ mm}$ to $1\text{ mm}$. This
potential is proposed to achieve intersection-free solutions, eliminate
nonphysical action at infinite distances, and maintain smoothness for better
numerics. We observe that this method again fits the framework
\eqref{eq:generic_force_law}, with a \emph{compliant} contact force law of the
form
\begin{equation*}
    f_n(\phi, v_n) = \kappa_\text{IPC}\,(\hat{d}-\phi)_+\left[\frac{(\hat{d}-\phi)_+}{\phi} - 2\ln(\phi/\hat{d})\right],
\end{equation*}
which is only non-zero for $\phi\in(0, \hat{d}]$. Performing Taylor expansion around
$\phi=\hat{d}$, we see that $f_n\approx
3\kappa_\text{IPC}(\hat{d}-\phi)_+^2/\hat{d}$, and the force models a quadratic spring of
stiffness $\kappa_\text{IPC}$ (with units of N/m). In the limit to
$\phi\rightarrow 0^+$, the force approximates
$f_n\approx\kappa_\text{IPC}\hat{d}^2/\phi$, the interior point force
\eqref{eq:logarithmic_barrier_force}.

In summary, this method models a thin, compliant layer around a rigid core
instead of the strict non-penetration conditions largely favored in the
literature. We do not view this as a flaw, as the authors have demonstrated their method's
effectiveness through extensive simulation studies. Instead, we see this as
an indication of the levels of rigidity that can be achieved in practice.

\subsection{Compliant Contact}

Previous sections demonstrate that even rigid approximations are effectively
compliant when analyzed in detail. This is precisely why this work embraces
compliance to develop a physics-based model that is transparent to users.

We consider a linear elastic law in the signed distance $\phi$ with Hunt \&
Crossley~\cite{bib:hunt_crossley} dissipation. As long as conditions
\eqref{eq:normal_sufficient_conditions} are met, other alternatives such as the
Hertz model can be used. In practice, however, users often find the linear model
satisfactory. We write this model within framework~\eqref{eq:generic_force_law}
as
\begin{equation}
    f_n(\phi, v_n) = k\,(-\phi)_+\,(1-dv_n)_+,
    \label{eq:linear_compliance_with_hunt_crossley}
\end{equation}
where $k$ is the linear contact stiffness and $d$ is the Hunt \&
Crossley dissipation parameter. This force is zero whenever $v_n \ge \hat{v}$,
with
\begin{equation}
    \hat{v}=\min(-\phi_0/\delta t, 1/d),
    \label{eq:breaking_velocity}
\end{equation}
the minimum normal velocity for contact to break. For $v_n < \hat{v}$ we define
the (indefinite) antiderivative $N^+(v_n)$
\begin{align*}
    N^+(v_n; \phi_0) &= \\
    \delta t\,k\,&\left[-v_n\left(\phi_0 + \frac{1}{2}\delta t\,v_n\right) + d\,\frac{v_n^2}{2}\left(\phi_0 + \frac{2}{3}\delta t\,v_n\right)\right].
\end{align*}

%However, we define the elastic contribution at the previous time step $f_{0} =
%-k\,\phi_0$ and write instead
%\begin{align*}
%    N^+(v_n; f_{0}) &= \\
%    &\delta t\left[v_n\left(f_0 + \frac{1}{2}\Delta f\right) - d\,\frac{v_n^2}{2}\left(f_0 + \frac{2}{3}\Delta f\right)\right],\nonumber
%\end{align*}
%with $\Delta f = -\delta t\,k\,v_n$. This form is numerically more stable to
%incorporate the modeling of continuous contact patches
%\cite{bib:masterjohn2021discrete} for which it is possible to have $k=0$ (or
%close to zero) for small face patches, but with a finite value of the product
%$f_{0} = -k\,x_0$.

Since the impulse is zero for $v_n \ge \hat{v}$, its antiderivative must be
constant. Therefore, we write
\begin{equation*}
    N(v_n) = N^+(\min(v_n, \hat{v}); f_{0}),
\end{equation*}
resulting in a continuous function for all values of $v_n$.

Though inherently compliant, this model effectively handles very stiff contact
due to the robustness of our convex formulation. In Section~\ref{sec:results},
we use Hertz theory to estimate steel stiffness, and
Section~\ref{sec:applications} demonstrates that our approach can manage
stiffnesses up to five orders of magnitude higher.

Finally, we observe that this model is easily extended to incorporate
\emph{hydroelastic contact}~\cite{bib:elandt2019pressure} --- a modern rendition
of \emph{Elastic Foundation}~\cite[\S 4.3]{bib:johnson1987_contact_mechanics} to
model continuous contact patches. We use the approach
in~\cite{bib:masterjohn2021discrete}.

\section{Contact with Friction}

\subsection{Regularized Friction}
\label{sec:regularized_friction}

As with normal impulses $\gamma_n(v_n)$, friction can be
modeled as a continuous function of state using a \emph{regularized}
approximation. Consider the model of isotropic friction
\begin{eqnarray}
    \bgamma_t(\vf{v}_c) &=&
    -\mu\,f(\Vert\vf{v}_t\Vert/\varepsilon_s)\,\gamma_n(v_n)\,\hat{\vf{t}}\nonumber\\
    f(s) &=& \frac{s}{\sqrt{1+s^2}}
    \label{eq:regularized_friction}\\
    \hat{\vf{t}} &=& \frac{\vf{v}_t}{\Vert\vf{v}_t\Vert}\nonumber
\end{eqnarray}
where the function $f(s)\le 1$ \emph{regularizes} Coulomb friction with
$\varepsilon_s$ the regularization parameter. When
$\Vert\vf{v}_t\Vert\ll\varepsilon_s$, the model behaves as viscous damping with
high viscosity. When $\Vert\vf{v}_t\Vert \gg \varepsilon_s$, the model
approximates Coulomb's law, with friction opposing slip velocity according to
the maximum dissipation principle and $\Vert\bgamma_t\Vert \rightarrow
\mu\gamma_n$. The choice of function $f(s)$ is somewhat arbitrary, as long as
$f(s) \le 1$ and $f(s)=0$ at $s=0$. We choose $f(s)$ such that
\eqref{eq:regularized_friction}~can be simplified to
\begin{eqnarray}
    \bgamma_t(\vf{v}_c) &=& -\mu\,\gamma_n(v_n)\,\hat{\vf{t}}_s\nonumber\\
    \hat{\vf{t}}_s &=&
    \frac{\vf{v}_t}{\sqrt{\Vert\vf{v}_t\Vert^2+\varepsilon_s^2}}
    \label{eq:regularized_friction_soft}
\end{eqnarray}
where we define the regularized or \emph{soft}
tangent vector $\hat{\vf{t}}_s$, which can be shown to be the gradient with
respect to $\vf{v}_t$ of the \emph{soft} norm
$\Vert\vf{x}\Vert_s=\sqrt{\Vert\vf{x}\Vert^2+\varepsilon_s^2} - \varepsilon_s$
(Appendix~\ref{sec:soft_norms}). Unlike $\hat{\vf{t}}$, which is not
well-defined at $\vf{v}_t=\vf{0}$, $\hat{\vf{t}}_s$ is well-defined and continuous
for all values of slip velocity. Moreover, \eqref{eq:regularized_friction_soft}
has continuous gradients, a  desirable property since it ensures the Hessian of
our convex formulation remains well-behaved, improving the convergence rate of
Newton iterations.

At this point, however, it is still unclear whether it is possible to incorporate such a
model of frictional contact into a convex formulation. We do know
SAP~\cite{bib:castro2022unconstrained} is a convex formulation of regularized friction
\begin{eqnarray}
    \bgamma_t(\vf{v}_c) &=& -\min\left(\frac{\Vert\vf{v}_t\Vert}{R_t}, \mu\gamma_n(v_n)\right)\hat{\vf{t}}
    \label{eq:sap_friction_regularization}
\end{eqnarray}
where $R_t$ is SAP's regularization parameter~\cite{bib:castro2022unconstrained}.
However, we seek a model superior to SAP's for two reasons
\begin{enumerate}
    \item SAP's linear model of dissipation can lead to severe artifacts at
    large dissipation values (see Section~\ref{sec:results}).
    \item  Although bounds on SAP's drift speed can be estimated, it is not possible to determine
    it precisely~\cite{bib:castro2022unconstrained}.
\end{enumerate}

In contrast, normal impulses of the form~\eqref{eq:discrete_normal_impulse}
allow the incorporation of physics-based models, such as the experimentally
validated Hunt \& Crossley model, and drift in \eqref{eq:regularized_friction}
is controlled precisely via $\varepsilon_s$. In the following sections, we
investigate the conditions for writing convex approximations of compliant models
\eqref{eq:discrete_normal_impulse} with regularized friction
\eqref{eq:regularized_friction}.

\subsection{Irrotational Contact Fields}

As is in~\cite{bib:castro2020}, we combine compliant contact with regularized
friction to write
\begin{eqnarray}
	\bgamma(\vf{v}_c) &=&
	\begin{bmatrix}
		-\mu\,n(v_n)\,\hat{\vf{t}}_s(\vf{v}_t)\\
		n(v_n)
	\end{bmatrix}.
    \label{eq:total_impulse}
\end{eqnarray}

However, this model is not necessarily the result of a potential and $\bgamma(\vf{v}_{c})
\ne -\partial\ell_{c}/\partial\vf{v}_{c}$ in general. In this section, we seek existence
conditions for $\ell_c(\mf{v})$ along additional conditions for convexity.

Helmholtz's theorem~\cite{bib:bhatia2012} states that any vector field admits
the decomposition into an irrotational field (zero curl) and a solenoidal field
(zero divergence)
\begin{eqnarray}
    \bgamma(\vf{v}_c) = -\frac{\partial\ell(\vf{v}_c)}{\partial\vf{v}_c} +
    \nabla\times\vf{A}(\vf{v}_c)
    \label{eq:helmholtz_decomposition}
\end{eqnarray}
with $\ell(\vf{v}_c)$ a scalar potential and $\vf{A}(\vf{v}_c)$ a \emph{vector
potential}.

This work ignores the solenoidal component and investigates \emph{irrotational
fields}, which satisfy the condition
\begin{equation}
    \nabla\times\bgamma = \mf{0}.
    \label{eq:gamma_curl}
\end{equation}

The normal component in~\eqref{eq:gamma_curl}
\begin{equation}
    \frac{\partial\gamma_{t,1}}{\partial v_{t,2}}=\frac{\partial\gamma_{t,2}}{\partial v_{t,1}},
    \label{eq:curl_normal}
\end{equation}
states that two-dimensional field $\bgamma_t(\vf{v}_t)$ is irrotational in the
$\vf{v}_t$ plane. Consider the generic isotropic friction model
\begin{equation}
    \bgamma_t=g(\Vert\mf{v}_t\Vert, v_n)\hat{\mf{t}}
    \label{eq:generic_friction_model}
\end{equation}
whose gradient is
\begin{equation*}
    \frac{\partial\bgamma_t}{\partial\mf{v}_t}=\frac{\partial g}{\partial \Vert\mf{v}_t\Vert}\mf{P}+g\frac{\mf{P}^\perp}{\Vert\mf{v}_t\Vert}
\end{equation*}
with symmetric projection matrices $\mf{P}$ and $\mf{P}^\perp$ (see
Appendix~\ref{sec:soft_norms}). Therefore, $\partial\bgamma_t/\partial\mf{v}_t$
is symmetric, and condition~\eqref{eq:curl_normal} is satisfied. Thus isotropic
friction fields are irrotational. While this work focuses on isotropic friction,
Appendix \ref{sec:anisotropic_friction} shows how to incorporate anisotropic
friction within this same framework.

Finally, the tangential components in~\eqref{eq:gamma_curl} lead to the
condition
\begin{equation}
    \frac{\partial\bgamma_t}{\partial v_n}=\frac{\partial\gamma_n}{\partial\mf{v}_t},
    \label{eq:curl_tangent}
\end{equation}
a condition generally not met by arbitrary contact models.
