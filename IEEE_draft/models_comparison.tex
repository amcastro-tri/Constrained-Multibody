\section{Comparative Analysis of the Models}
\label{sec:comparative_analysis}

All models in Section \ref{sec:models_family} are convex approximations of
contact, each with its own strengths and limitations. Hereinafter, we refer to
each of these formulations to as SAP \cite{bib:castro2022unconstrained},
\emph{Lagged} (section \ref{sec:lagged_model}), and \emph{Similar} (section
\ref{sec:similar_model}). We summarize their properties and artifacts in Table
\ref{tab:models_properties} and analyze them in detail in the following
subsections. Artifacts are a result of the convex approximation, and are only
present during slip, not in stiction.  Lagged does not introduce artifacts.

\begin{table*}[h]
    \centering
    \begin{tabular}{c|c|c|c|cc|c}
                                     Model & Consistent  & Strongly coupled &
                                     Gliding during slip   & \multicolumn{2}{|c|}{Compliance
                                     modulation} & Transition slip \\
                                     & & & & Stiffness & Dissipation \\
        \hline
        SAP           & No  & Yes & $\mu(\delta t + \tau_d)\Vert\vf{v}_t\Vert$ &
        $k/(1+\tilde{\mu}^2)$ &$\tau_d$ & $\sigma\,\text{w}\,\mu\,\gamma_n$\\
        Lagged        & Yes & No  &  ---                                       &
        --- & --- & $v_s$ \\
        Similar       & No  & Yes & $\mu\delta t\Vert\vf{v}_t\Vert$            &
        $k\,(1+\mu\,d\,\Vert\vf{v}_t\Vert)$ & $d/(1+\mu\,d\,\Vert\vf{v}_t\Vert)$
        & $v_s$ \\
    \end{tabular}
    \caption{Properties of each convex approximation in Section \ref{sec:models_family}.}
    \label{tab:models_properties}
\end{table*}

\subsection{Gliding During Slip}
\label{sec:gliding_artifact}

SAP, as well as the formulations from Anitescu and Todorov
\cite{bib:anitescu2006,bib:todorov2011}, introduces a number of artifacts as the result
of the convex approximation of contact. The most well-known artifact is that
objects \emph{glide} at a finite offset during slip \cite{bib:mazhar2014,
bib:horak2019, bib:castro2022unconstrained}. For SAP,
this offset is \cite{bib:castro2022unconstrained}
\begin{equation}
    \phi_\text{offset} = \mu(\delta t + \tau_d)\Vert\vf{v}_t\Vert.
    \label{eq:sap_gliding_offset}
\end{equation}
With zero dissipation ($\tau_d=0$), this reduces to the gliding distance $\mu\delta
t\Vert\vf{v}_t\Vert$ in Anitescu's formulation of rigid contact
\cite{bib:anitescu2006}. There are no artifacts during stiction.

While the term $\mu\delta t\Vert\vf{v}_t\Vert$ can be negligible for small time
steps and slip speeds, the term $\mu\tau_d\Vert\vf{v}_t\Vert$ can dominate when
dissipation is large. Therefore, users must face the trade-off between accurate
dissipation modeling and the magnitude of this gliding artifact, making the
choice of this parameter cumbersome.

The Similar approximation eliminates this coupling with dissipation,
though it still introduces the offset $\mu\delta t\Vert\vf{v}_t\Vert$. We see
this by replacing $v_n$ with $z$ from \eqref{eq:similar_grouping} into
\eqref{eq:linear_compliance_with_hunt_crossley}. This is desirable as gliding
vanishes as the time step size is reduced.

The Lagged approximation does not introduce gliding.

\subsection{Compliance Modulation}
\label{sec:compliance_modulation}

For SAP, the effective compliance during slip is $k_\text{eff} =
k/(1+\tilde{\mu}^2)$, where $\tilde{\mu}=\mu(R_t/R_n)^{1/2}$
\cite{bib:castro2022unconstrained} depends on the ratio of the tangential to
normal regularization parameters. SAP chooses this ratio so that
$\tilde{\mu}\approx0$, but it is still non-zero. We refer to this artifact as
\emph{compliance modulation}, and it was first reported in
\cite{bib:castro2022unconstrained}.

Replacing $v_n$ with $z$ in \eqref{eq:linear_compliance_with_hunt_crossley} and
factoring out the term $1+\mu\,d\,\Vert\vf{v}_t\Vert$, we see that during slip,
the Similar approximation has effective stiffness $k_\text{eff} =
k\,(1+\mu\,d\,\Vert\vf{v}_t\Vert)$ and dissipation $d_\text{eff} =
d/(1+\mu\,d\,\Vert\vf{v}_t\Vert)$. That is, the effective stiffness increases
while the effective dissipation decreases during slip.

The Lagged approximation does not introduce compliance modulation.

\subsection{Strong Coupling}
\label{sec:strong_coupling}

In the SAP and Similar models, Coulomb's law tightly couples friction and normal
components, while the Lagged model introduces a lag in the normal force. Our
numerical studies (Sections \ref{sec:results} and \ref{sec:applications}) show
that this lag effect is noticeable only in high-energy impacts where normal
forces change rapidly. Otherwise, the approximation is $\mathcal{O}(\delta t)$,
consistent with the first-order Taylor expansion of the signed distance
introduced in Section \ref{sec:frictionless_contact}. Moreover, Our grasp
stability analysis (Section \ref{sec:franka_hand}) confirms that the lagged
approximation is suitable even for highly dynamic manipulation tasks.

\subsection{Consistency}
\label{sec:consistency}

In the context of ordinary differential equations (ODEs), a scheme is
\emph{consistent} when the original set of ODEs is recovered in the limit $\delta
t\rightarrow 0$. This is not the case for SAP and Similar models, due to the
existence of \emph{compliance modulation} (Section
\ref{sec:compliance_modulation}) regardless of the time step size. In particular
for SAP, the gliding artifact does not vanish in \eqref{eq:sap_gliding_offset}
unless dissipation is zero. Conversely, Lagged is a consistent first order
approximation of \eqref{eq:total_impulse}.

\subsection{Stick-Slip Transition}
\label{sec:stiction_tolerance}

Since the models in this work \emph{regularize} friction, transition
between stiction and sliding occurs at a finite slip velocity. For Lagged and
Similar models, this transition is parameterized by $\varepsilon_s$ in
\eqref{eq:regularized_friction}, and it is fixed by the user to a value $v_s$
which we call the \emph{stiction tolerance}.

For SAP, we see from \eqref{eq:sap_friction_regularization} that stick-slip
transitions occur at a magnitude of slip equal to
\begin{equation}
    v_s = \sigma\,\text{w}\,\mu\,\gamma_n,
    \label{eq:sap_stiction_tolerance}
\end{equation}
where we used the fact that SAP uses $R_t = \sigma\,\text{w}$, with $\text{w}$ a
diagonal approximation of the Delassus operator
\cite{bib:castro2022unconstrained} (the \emph{effective inverse mass} of the
contact). The dimensionless parameter $\sigma$ is set to $\sigma=10^{-3}$, a
good trade-off between numerical conditioning and a tight approximation of
stiction, as required for the simulation of manipulation applications in
robotics. Refer to \cite{bib:castro2022unconstrained} for a thorough study of
SAP's properties.

\section{Impacts, Stiffness and Conditioning}
\label{sec:impacts_and_conditioning}

We are interested in the numerical \emph{stiffness} introduced by the
regularization of friction as it is critical for a robust numerical
implementation. We define $G_t= d\Vert\bgamma_t\Vert/d\Vert\vf{v}_t\Vert$ and
examine its value at $\Vert\vf{v}_t\Vert=0$. For all approximations we find
\begin{equation*}
    G_t = \frac{\mu}{v_s}\tilde{\gamma}_n,
\end{equation*}
with $v_s$ the stiction tolerance (Table \ref{tab:models_properties}),
$\tilde{\gamma}_n =\gamma_n$ for SAP and Similar models, and $\tilde{\gamma}_n
=\gamma_{n,0}$ for Lagged.

As seen in \eqref{eq:sap_stiction_tolerance}, SAP's stiction tolerance depends
on impulse, often much higher during impacts than in sustained contact
(e.g., a robot grasping an object). This effectively increases SAP's
regularization during impacts, making the problem better-conditioned but
reducing stiction accuracy. In contrast, Lagged and Similar solve a much stiffer
problem given the stiction tolerance $v_s$ is constant. We confirm this in
Section \ref{sec:clutter} in a case with impacts.

The situation reverses during sustained contact, as the normal impulse depends
mainly on object's weight, and external forces, not rapid velocity changes.
Since impulse scales with time step size, $G_t = \mathcal{O}(\delta t)$ for
Similar and Lagged, while SAP's $G_t = 1/R_t$ remains constant (with $R_t$ from
\eqref{eq:sap_friction_regularization}), leading to a stiffer problem. This is
confirmed in Section \ref{sec:clutter} as we analyze the numerical conditioning
of the problem and solver's number of iterations.

It could be argued that, for most robotics applications,
accurately resolving stiction during impacts is unnecessary. Therefore, we
propose a \emph{regularized} version of the Lagged model, where the
regularization parameter is
\begin{equation}
    \varepsilon_s = \max(v_s, \sigma\,\text{w}\,\mu\,\gamma_{n, 0}).
    \label{eq:regularized_vs}
\end{equation}
In this formulation, regularization is \emph{softened} during strong impacts as
values of $\gamma_{n, 0}$ are higher, but a tight value bounded by $v_s$ is used
for sustained contacts. We study the effect of this formulation in Section
\ref{sec:clutter}.
