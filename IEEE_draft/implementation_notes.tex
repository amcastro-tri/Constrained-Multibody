\section{Implementation}
\label{sec:implementation}

Our work is implemented in Drake \cite{bib:drake}, a robotics toolkit that
provides modeling abstractions and optimization tools for the modeling,
simulation, and analysis of robotics systems. Our implementation includes support
for deformable Finite Element Models (FEM), holonomic constraints, PD
controllers with effort limits, reflected inertia, and the modeling of
continuous contact patches with \emph{Hydroelastic Contact}. Refer to our
previous work for further details
\cite{bib:castro2022unconstrained,bib:masterjohn2021discrete,bib:han2023}.

SAP \cite{bib:castro2022unconstrained} uses Newton's method to compute a search
direction $\Delta\mf{v}$ for \eqref{eq:unconstrained_fomulation} according to
\begin{equation}
    \mf{H}(\mf{v})\Delta\mf{v} = -\mf{r}(\mf{v}),
    \label{eq:search_direction}
\end{equation}
where $\mf{r}(\mf{v})=\partial\ell/\partial\mf{v}$ is the \emph{residual} of the
primal cost $\ell(\mf{v})$, and $\mf{H}(\mf{v})$ is its Hessian. Upon convergence,
the residual $\mf{r}(\mf{v})$ is zero, which is equivalent to the linearized
balance of momentum \eqref{eq:optimality_condition}. The solution at iteration
$m$ is updated as
\begin{equation*}
    \mf{v}_{m+1} = \mf{v}_{m} + \alpha\Delta\mf{v},
\end{equation*}
with $\alpha$ determined via a line search along $\Delta\mf{v}$ to minimize
$\ell(\mf{v})$. Our line search uses a Newton-Raphson method based on \cite[\S
9.4]{bib:numerical_recipes}, augmented with bracketing and bisection to
guarantee convergence. The derivatives required for the line search are computed with
$\mathcal{O}(n)$ complexity \cite{bib:castro2022unconstrained}, enabling
convergence to machine precision with negligible computational overhead.
This strategy has proven to be very robust in practice.

Multibody \emph{tree} structures create distinct cliques, while degrees of
freedom (DOFs) for modeling a deformable body are also grouped into their own
cliques \cite{bib:han2023}. Similarly, constraints involving the same pair of
cliques are grouped into \emph{clusters}. We exploit this structure using a
supernodal Cholesky factorization \cite[\S 9]{bib:davis2016survey} of the
Hessian $\mf{H}$ in \eqref{eq:search_direction} that takes advantage of dense
algebra optimizations. We compute the elimination ordering of the supernodes using
approximate minimum degree (AMD) ordering \cite{bib:amestoy1996approximate} to
minimize fill-ins.

\subsection{Differentiation Through Contact}
\label{sec:gradients}

Our implementation in Drake provides an end-to-end solution for the computation
of gradients through contact for applications such as system identification,
reinforcement learning, and trajectory optimization. We implement a novel
\emph{hybrid} approach that utilizes automatic differentiation to propagate
gradients of the Featherstone's dynamics and geometry through the SAP
formulation via the \emph{implicit function theorem}.

We denote differentiation parameters with $\btheta\in\mathbb{R}^{n_\theta}$,
which can include physical quantities such as mass and inertias, contact
parameters, actuation, and even the previous state of the multibody system.
Using this notation, SAP's optimality condition can be written in terms of the
residual in \eqref{eq:search_direction} as $\mf{r}(\mf{v}; \btheta) = \mf{0}$.
In this notation, the residual is a function of the
generalized velocities $\mf{v}$ \emph{given} a set of parameters $\btheta$.

We use the implicit function theorem on $\mf{r}(\mf{v}; \btheta) = \mf{0}$ 
\begin{equation*}
    \frac{\partial \mf{r}(\mf{v}; \btheta)}{\partial\mf{v}}\frac{d\mf{v}}{d\btheta} + \frac{\partial \mf{r}(\mf{v}; \btheta)}{\partial\btheta} = \mf{0},
\end{equation*}
and note that $\partial\mf{r}/\partial\mf{v}=\mf{H}(\mf{v})$ to write
\begin{equation}
    \mf{H}\frac{d\mf{v}}{d\btheta} = -\frac{\partial \mf{r}(\mf{v}; \btheta)}{\partial\btheta}.
    \label{eq:v_derivatives}
\end{equation}

In our approach, the expensive-to-compute Cholesky factorization of $\mf{H}$ is
only computed at each Newton iteration in \eqref{eq:search_direction} during
forward simulation. Upon convergence, $\mf{H}$ is already assembled and
factorized and is thus reused in \eqref{eq:v_derivatives} an additional
$n_\theta$ times to propagate derivatives through the solver into gradients
$d\mf{v}/d\btheta$ of the generalized velocities.

In our hybrid approach, $\partial\mf{r}/\partial\btheta$ in
\eqref{eq:v_derivatives} is computed with automatic differentiation. This
enables the computation of gradients through arbitrarily complex geometric
models, while the implicit function theorem propagates derivatives accurately
and efficiently through the contact resolution phase.
