\section{Introduction}

\IEEEPARstart{S}{imulation} of multibody systems with frictional contact is
essential in robotics, aiding hardware optimization, controller design and
testing, continuous integration, and data generation for machine learning.
Accurate physics models enable advanced controller design and trajectory
optimization algorithms. However, achieving robust, accurate, and efficient
simulations for contact-rich robotics applications remains challenging.

Rigid body dynamics with frictional contact are complicated by non-smooth
solutions. Acceleration-level formulations can lead to singular configurations,
known as Painlev\'e paradoxes, where solutions may not exist. Discrete
velocity-level formulations circumvent this by allowing discrete velocity jumps and
impulsive forces. While many variants exist, the general form of a discrete
formulation enforces balance of momentum subject to contact constraints,
with additional constraints to incorporate Coulomb's law of friction under the
maximum dissipation principle. With the addition of decision
variables and Lagrange multipliers, the result is a large and challenging-to-solve
Non-linear Complementarity Problem (NCP) with a much larger number of
variables than the original problem.

Solving NCPs robustly and efficiently has remained elusive. NCPs are equivalent
to non-convex global optimization problems, which are generally
NP-hard~\cite{bib:Kaufman2008}. Therefore, NCPs may lack solutions or have
multiple solutions. In practice, iterative solvers might incorrectly assume
convergence or terminate early, leading to solutions that fail to satisfy the
original equations or violate physical laws.

There are other, less frequently-discussed issues --- numerical conditioning
and implementation details --- that affect convergence properties and robustness. These
are important problems when it comes to the simulation of complex systems, with
either many degrees of freedom (DOFs), a large number of constraints,
or both. Solutions that work for small systems often do not work when faced
with real-world engineering applications.

In an attempt to make the problem tractable, Anitescu introduced a convex
approximation of contact~\cite{bib:anitescu2006}, effectively replacing
the original NCP with a convex approximation that guarantees the existence of solutions.
Later, Todorov~\cite{bib:todorov2011} regularized this formulation to write a strictly
convex formulation with a unique solution. Our previous work~\cite{bib:castro2022unconstrained}
introduced the convex Semi-Analytic Primal
(SAP) formulation for modeling compliant contact. SAP embraces compliance
to provide a robust and performant tool that targets robotics applications ---
where grippers feature compliant surfaces for stable grasps, and robotic feet
are often padded with compliant materials. Although the formulation is
inherently compliant, rigid contact can be modeled to good approximation using a
\emph{near-rigid} approximation. The work focused on physical correctness,
numerical conditioning, and robustness.

However, the SAP formulation has a number of limitations. The most well-known is
the artifact of \emph{gliding} during slip, an artifact inherited from the
original formulation by Anitescu~\cite{bib:mazhar2014} and shared by Todorov's
formulation~\cite{bib:todorov2011}. In Anitescu's formulation, as objects slip, they
glide at a finite distance $\delta t\mu\Vert\vf{v}_t\Vert$, proportional to the
time step $\delta t$, the coefficient of friction $\mu$, and the slip speed
$\Vert\vf{v}_t\Vert$. While the artifact vanishes as the time step approaches zero,
its effect can be significant at the large time step sizes required for interactive
simulations. Moreover, we show that both SAP's compliant model and
Todorov's regularized model are inconsistent --- the gliding effect is also
proportional to dissipation and does not vanish as time step goes to zero. This
is nonphysical and makes parameter selection a cumbersome trade-off between
physical accuracy and the magnitude of the gliding artifact. Finally, SAP's
model of compliance is intrinsic to its convex formulation, and therefore it is
not possible to incorporate other well-known and experimentally validated models
of contact, such as Hertz and Hunt \& Crossley~\cite{bib:hunt_crossley}.

Still, the SAP formulation provides a robust solution with theoretical
convergence guarantees that transfer to a practical software implementation.
Therefore, the objective of this work is to extend the SAP formulation to reduce
and even eliminate many of these limitations, while preserving its
convexity along with its convergence and robustness guarantees.
