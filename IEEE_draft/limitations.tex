\section{Conclusions}
\label{sec:conclusions}

We presented a novel theory for the convex approximation of contact. Our
mathematical framework establishes a family of convex approximations of
frictional contact, and we show that previous approaches
\cite{bib:anitescu2006,bib:todorov2011,bib:castro2022unconstrained} are members of
this family. This framework enables us to incorporate complex physics-based
models of contact, such as the Hunt \& Crossley \cite{bib:hunt_crossley} model,
within a convex formulation for the first time. These models, grounded
in physics and experimentally validated, have the potential to narrow the
\emph{sim2real} gap. Within this framework, we develop two convex approximations
of regularized friction: \emph{Similar} and \emph{Lagged}.

This work presents a thorough characterization of these approximations in
terms of consistency, the coupling between normal and tangential components, and
artifacts introduced by the convex approximation. While previous work has documented
\emph{gliding} during slip over a distance of $\delta t\mu\Vert\vf{v}_t\Vert$, we
identify previously unrecognized artifact characteristics of compliant contact
\cite{bib:todorov2011,bib:castro2022unconstrained}.
We validate these findings with a rich set of test cases designed to expose these
problems and gain insight into the new formulations. Moreover, our analysis
led to new understandings of the coupling between normal and frictional
components of the contact, allowing us to design a \emph{regularized} scheme that improves
numerical conditioning during difficult to resolve impact events. 

Our investigation concludes that, even though normal and friction forces are
weakly coupled, the Lagged approximation is well-suited for the modeling of most
robotic tasks. Moreover, the Lagged approximation completely eliminates
artifacts associated with previous convex approximations.

Our work is implemented in the open-source robotics toolkit Drake \cite{bib:drake}. 
We rigorously tested our implementation on various robotics-relevant problems, including
an iLQR application to highlight our differentiable pipeline and a deformable body
simulation to demonstrate compatibility with FEM-based methods.

One of the most significant limitations of our simulation pipeline is
related to \emph{tunneling} or \emph{passthrough} problems, where objects can
bypass each other without registering contact due to the nature of discrete contact detection.
This issue is particularly pronounced
with large time step sizes and thin objects. We are currently investigating a solution
based on \emph{speculative constraints} \cite{bib:catto} for hydroelastic contact
\cite{bib:elandt2019pressure,bib:masterjohn2021discrete} that we believe can
help mitigate this issue.
