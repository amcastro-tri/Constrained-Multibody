\section{Novel Contributions}

We seek to mitigate nonphysical artifacts introduced by by existing convex
approximations \cite{bib:anitescu2006, bib:todorov2011,
bib:castro2022unconstrained}, incorporate experimentally validated contact
models, increase the fidelity of frictional contact, and improve numerical
performance.

We extend SAP \cite{bib:castro2022unconstrained} by introducing incremental
potentials~\cite{bib:pandolfi2002}. Previous work introduces Incremental
Potential Contact (IPC)~\cite{bib:li2020ipc}, a method that guarantees
intersection-free solutions. The authors demonstrate their method in a large
variety of simulation cases. However, like NCPs, this non-convex formulation can
either have no solution (the feasible set is empty) or fall into a local minimum
that does not obey the law of physics.

In this work, we develop conditions to design convex incremental potentials
that can be incorporated into our SAP~\cite{bib:castro2020} formulation. To the
knowledge of the authors, conditions for convexity in this context have never
been explored.

We summarize our novel contributions as follows:
\begin{enumerate}
    \item We introduce a framework to generate convex approximations of
    arbitrarily complex contact models.
    \item We present necessary conditions for existence and convexity of an
    incremental potential.
    \item We show how to incorporate Coulomb friction independent of the
    complexity of the compliant contact model. Our framework supports arbitrary
    forms of regularization, facilitating the design of accurate models with
    desirable numerical properties.
    \item We present the first convex approximation of compliant
    contact with Hunt \& Crossly dissipation~\cite{bib:hunt_crossley}.
    \item We show that the same framework can be used to incorporate barrier
    functions to model rigid contact.
    \item In addition to SAP, we develop two new formulations of contact with
    dry friction within this same framework.
\end{enumerate}

In addition to these main research contributions, our work includes a fully
differentiable open-source implementation in Drake~\cite{bib:drake}. We compare
models in Section~\ref{sec:comparative_analysis}, validate them in
Section~\ref{sec:results}, and stress test them with engineering applications
cases in Section~\ref{sec:applications}.
